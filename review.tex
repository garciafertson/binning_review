\documentclass{article}
\usepackage[hidelinks]{hyperref}
\usepackage{apacite}
\usepackage{graphicx}
\usepackage{a4wide}
\usepackage{booktabs}
\usepackage{rotating}
\bibliographystyle{apacite}

\title{Metagenomic Binning Pipelines - the State of the Art}
\date{}
%\author{Theo Portlock}
\begin{document}
\maketitle

\section{Abstract}
\begin{itemize}
	\item \emph{Decision tree graphical abstract for the choice of binning algorithm}
	\item \emph{Features that distinguish binning algorithms}
	\item \emph{Some guidelines for choosing the correct binning techniques appropriate for a given study}
\end{itemize}
New generations of sequencing platforms coupled to numerous bioinformatics tools have led to rapid technological progress in metagenomics and metatranscriptomics to investigate complex microorganism communities.
Nevertheless, a combination of different bioinformatic tools remains necessary to draw conclusions out of microbiota studies.
As sequencing costs have dropped at a rate above 'Moore's law', bigger data sets are available, and proportional costs of analysis have risen as a consequence.
Binning is the grouping of assembled metagenomic contigs by their genome of origin.
Algorithms for binning are a rapidly evolving field.
The number of these algorithms are growing over time.
Selecting the most appropriate binning algorithm can be a daunting task and is influenced on computational resources available and experimental variables relating to the sequencing.
This review serves as a roadmap to direct the researcher to the binning algorithm that best suits their needs.


\section{Background}
\begin{itemize}
	\item \emph{General introduction to meaning of binning, include MSP in definition}
	\item \emph{increase in popularity of the field of metagenomics}
	\item \emph{Talk about reduced cost in sequencing and scheduling efficiency in pipelines}
	\item \emph{Talk about HMP}
	\item \emph{New emerging technologies/ field in development}
	\item \emph{Aims of the review}
	\item \emph{}
\end{itemize}

The main purpose of binning tools is to reconstruct species/biological entities from metagenomic samples.  

Compared to amplicon, shotgun metagenome can provide functional gene profiles directly and reach a much higher resolution of taxonomic annotation.
However, due to the high demands on computational resources, cost, and expertise necessary to perform this analysis, this method has been used
The substantial decrease in cost of sequencing however has lessened this burden. 
At the time of writing, shotgun metagenomic sequeng costs on average three times as much as 16S sequencing comparitavely.
A review on the benchmarking binning algorithms was done by \citeNP{yue2020evaluating}.
The objectives of this review is for the reader to be better informed about the latest algorithms (since 2017) for binning metagenomic samples
The second part of this review is for the reader to be informed about distinguishing factors between the methods
The last part is for the reader to make an informed decision based on those factors for their needs
Difference between MAG and MSP
General Weaknesses of binning algorithms - Future directions
%Sections
\begin{itemize}
	\item \emph{List of recent algorithms - table - technologies - general specialisation}
	\item \emph{Classify binning algorithms based on their objectives, guideline for algorithm choice, subsection msp mag }
	\item \emph{Current limitations and Future directions}
\end{itemize}


\section{Methods for metagenomic binning (list of methods since 2017 topic of development, field of improvements) }
\begin{itemize}
	\item \emph{Not sure if to focus on this or the appropriateness}
	\item \emph{The increased impact of machine learning in analysis}
	\item \emph{Short section - just for past-present-future completeness}
\end{itemize}
\subsection{Recently developed algorithsm on binning metagenomic sequences}

A metagenomic sample is comformed of many organisms and the standard procedure is to retrieve the sequences from the mixture of organisms. The
final goal of binning is to reconstruct the sequnces from each organism present in the original sample.

Among the binning tools developed in recent years we can distinguish a subset dedicated to cluster reads (read-binning) (MetaBBC-LR, BioBloom Tools, CLAME, LVQ-KKN, Meta VW, HirBin, MEGAN-LR)
The main purpose of read-binning tools is to preprocess reads into clusters for a posterior targeted assemby, here we find reference-free and non-reference-free tools, and tools designed for short-read or long-read sequencing technologies

The majority of binning tools we can find are oriented toward clustering contigs (contig-binning) into bins, which may represent the genome from a single biological entity/organism.
Contig-binning tools normally rely on coverage information and sequence composition. Progress in contig-binning algorithms can be seen in the proposals to integrate new sources of information (for example, from scaffold-graphs(Binnacle), paired-end reads(COCACOLA), or 3D contact information(MetaTOR)) and state of the art algorithms in machine learning (CoCoNet,VAMB)


\subsection{Metagenome Assembled Genomes}
A Metagenome-Assembled Genome (MAG) is a single-taxon assembly based on one or more binned metagenomes that has been asserted to be a close representation to an actual individual genome (that could match an already existing isolate or represent a novel isolate).


\subsection{Binning microbial genomes with deep learning}
The integration of deep learning techniques into the field of metagenomics has revolutionised the field of metagenomics.
The VAMB pipeline was developed to take advantage of variational autoencoders; a generative machine learning model that uses a combination 
Improved metagenome binning and assembly using deep variational autoencoders
Nature biotechnology - 4th Jan 2021
the VAMB pipeline \cite{nissenimproved}

\subsection{Binning for viral genomes}
New insights from uncultivated genomes of the global human gut microbiome
Nature - 13th March 2019 \cite{nayfach2019new}


\section{Chosing the most appropriate binning algorithm (Classification by output)}
\begin{sidewaystable}
\begin{tiny}
\centering
\caption[Comparison of binning algorithms]{Comparison of binning algorithms}
	\begin{tabular}{p{2cm}|p{0.6cm}|p{4cm}|p{4cm}|p{3.4cm}}
\toprule
                           Software/Algorithm &  Year &                                                                       Description/purpose &                                                               Comment/Highlight &                            Doi \\
\midrule
      CoCoNet \cite{arisdakessian2021coconet} &  2021 &                                           Deep learning tool for Viral Metagenome Binning &                                                  Reconstuction of viral genomes & 10.1093/bioinformatics/btab213 \\
     Binnacle \cite{Muralidharah2021binnacle} &  2021 &                                        Using scaffolds to improve Metagenomic bin quality &                                               Incorporates scaffold information &      10.3389/fmicb.2021.638561 \\
                   VAMB \cite{nissenimproved} &  2021 &                                    Metagenome binning using deep variational autoencoders &                                         Autoencoder algorithm, fast processing  &     10.1038/s41587-020-00777-4 \\
      phyloFlashi \cite{Harald2020phyloflash} &  2020 &                                                         ssrRNA profiling and MAG assembly &                     Incorporates ssrRNA profiling information into MAG assembly &      10.1128/mSystems.00920-20 \\
 MetaBCC-LR \cite{wickramarachchi2020metabcc} &  2020 &                                                        Metagenomic binning for Long-Reads &                                   Suitable for long reads sequencing technology & 10.1093/bioinformatics/btaa441 \\
            BioBloom \cite{ToolsChu2020bloom} &  2020 &                                     Binning of reads for alignment free targeted assembly &                        Data preparation for targeted assembly using space seeds &        10.1073/pnas.1903436117 \\
  GraphBin \cite{mallawaarachchi2020graphbin} &  2020 &                              Refined binning of metagenomic contigs using assembly graphs &                                         Incorporates assembly graph information & 10.1093/bioinformatics/btaa180 \\
               MetaCon \cite{Qian2019metacon} &  2019 &        Unsupervised binning k-mers and coverage focussing on contigs of different lenghts &                                               Focusses different contig lengths &      10.1186/s12859-019-2904-4 \\
                 VirBin \cite{Chen2019virbin} &  2019 &                                           Binning viral haplotypes from assembled contigs &                                                           Viral haplotypes MAGs &      10.1186/s12859-019-3138-1 \\
                  MAGO \cite{murovec2019mago} &  2019 &                                             Framework for Production and analysis of MAGs &                                                             Integrated pipeline &          10.1093/molbev/msz237 \\
             SeqDex \cite{chiodi2019seqdechi} &  2019 &                          Genome separation of Endosymbionts from mixed sequencing samples &                                                  Identification of endosymbiont &       10.3389/fgene.2019.00853 \\
    SqueezeMeta \cite{tamames2019squeezemeta} &  2019 &                    Pipeline covering the complete metagenomic/metatranscriptomic analysis & Internal checks that provides information about consistency of contigs and bins &       10.3389/fmicb.2018.03349 \\
             MetaTOR \cite{Baudry2019Metator} &  2019 &                       High quality MAG binning from mammalian guts using meta3C libraries &                                             Incorporates 3D contact information &       10.3389/fgene.2019.00753 \\
         MetaBAT (v2) \cite{Kang2019metabat2} &  2019 &        Adatptative binning algorithm for genome reconstruction from metagenome assemblies &                        Eliminates manual parameter tuning from previous version &             10.7717/peerj.7359 \\
                 MetaBMF \cite{Ma2019MetaBMF} &  2019 &   Scalable binning algorithm for species and strain level large scale metagenomic studies &                                               Applicable to large scale studies &  10.1093/bioinformatics/btz577 \\
             SolidBin \cite{wang2019SolidBin} &  2019 &                          Improving metagenome binning with semi-supervised normalized cut &                                                 Improved clustering performance &  10.1093/bioinformatics/btz253 \\
           Autometa \cite{miller2019autometa} &  2019 &                       Extraction of microbial genomes from individual shotgun metagenomes &                                                Handles eukaryotic contamination &             10.1093/nar/gkz148 \\
              CLAME \cite{benavides2018clame} &  2018 &                             Aligment based algorithm allowed description of novel species &                                                     Aligment based read binning &      10.1186/s12864-018-5191-y \\
                   LVQ-KNN \cite{belka2018lv} &  2018 &                                   Composition based RNA or DNA binning of short sequences &                                         Classification into DNA or RNA sequence & 10.1016/j.virusres.2018.10.002 \\
            MSPminer \cite{plaza2019mspminer} &  2018 &             Abundance based reconstitution of microbial pan-genomes from metagenomic data &                                                       Pan genome reconstitution &  10.1093/bioinformatics/bty830 \\
         MetaWRAP \cite{Uritskiy2018metawrap} &  2018 &                           Flexible pipeline for genome resolved metagenomic data analysis &                                                 Hybrid bin extraction algorithm &      10.1186/s40168-018-0541-1 \\
              Meta VW \cite{vervier2016large} &  2018 &                                      Large scale Machine Learning Sequence classification &                                Machine learning for reads based on Kmer profile &    10.1007/978-1-4939-8561-6_2 \\
                     BMC3C \cite{Yu2018BMC3C} &  2018 &                  Binning contigs using codon usage sequence composition and read coverage &                                                    Adds codon usage information &  10.1093/bioinformatics/bty519 \\
           DAS Tool \cite{sieber2018recovery} &  2018 &                                            Dereplication aggregation and scoring strategy &                                      Combines several binning algorithm results &      10.1038/s41564-018-0171-1 \\
               MEGAN-LR \cite{huson2018megan} &  2018 &                                                       Long read/contig taxonomic binning  &                              Aligment of long reads against reference sequences &      10.1186/s13062-018-0208-7 \\
                 CoMet \cite{herath2017comet} &  2018 &                                           Binning workflow using coverage and composition &                          Single sample, includes GC content and 4-mer frequency &      10.1186/s12859-017-1967-3 \\
               MetaGen \cite{Xing2017MetaGen} &  2017 &                                 Reference-free learning with multiple metagenomic samples &                                                       Requires multiple samples &      10.1186/s13059-017-1323-y \\
         BusyBee Web \cite{Laczny2017BusyBee} &  2017 &                                            Bootstrapped supervised binning and annotation &                                                              Supervised binning &             10.1093/nar/gkx348 \\
            ICoVer \cite{Broeksema2017IcoVer} &  2017 &        Interactive visualisation tool for verification and refinement of metagenomic bins &                                                  Interactive visualisation tool &      10.1186/s12859-017-1653-5 \\
         BinSanity \cite{Graham2017Binsanity} &  2017 &                           Unsupervised clustering using coverage and affinity propagation &                                             Reduce bias for high/low abundance  &             10.7717/peerj.3035 \\
Binning refiner \cite{Song2017Binningrefiner} &  2017 &              Improved genome bins through the combination of different binning algorithms &                                     Combination of different binning algorithms &  10.1093/bioinformatics/btx086 \\
               COCACOLA \cite{Lu2016COCACOLA} &  2016 & Binning contigs using composition, read coverage, co-aligment and paired end read linkage &                                 Adds paired end read and coaligment information &  10.1093/bioinformatics/btw290 \\
   GroopM (v2:2017) \cite{Imelfort2014GroopM} &  2014 &                Tool for automatic recovery of population genomes from related metagenomes &              Adds differential coverage to complement composition based binning &              10.7717/peerj.603 \\
\bottomrule
\end{tabular}

\label{Tbinningsoftware}
\end{tiny}
\end{sidewaystable}

Resource management is an important factor in the choice of binning algorithm.
The tradeoff between number of CPU's, memory, and time are important considerations.
Pipeline vs standalone?
Alignment based or alignment free
An analysis pipeline is defined as a program that combines several programs in a defined order to complete a complex analysis.
Improperly developed, validated, and/or monitored pipelines may generate inaccurate results.

\subsection{MSPs, binning co-abundant genes}

Binning of co-abuntant genes represents an alternative proposal to reconstruct species/biological entities from a set of metagenomic samples.

Co-abundant gene binning methods assume each gene coming from a shared chromosome will display proportional abundances across samples, if you have enough samples from a common enviroment you can identify the sets of genes from a common organism of origin (MLGs Chameleon-clust 2012, CAGs and MGSs Canopy 2014, Markovclust-MGCs Karlsson 2013, MSPs MSPminner 2018).

To the extent of our knwoledge, in the past few years MSPminer is the only available Software exploiting this approach. MSPminer introduced a robust proportionality measure detecting co abundant but no necesarily co ocurring. This tools groups co-abundant genes into Metagenomic Species Pan-genomes or MSPs and classify genes within an MSP as core, accesory and shared.  

The factors that impact directly on MSPs quality include the sample composition, the sequencing depth, the previos bioinforamtic steps to build the reference gene dataset and to map the reads.
A high number of samples with varying phenotipes improve the quality of MSPs.

MSPs can be employed for taxonomic profiles of new samples from similar ecosystems, to compare strains between samples building a presence/absence table of accesory genes and for biomarker discovery. By binning contigs carrying genes from the same MSP it is also possible to build a MAG.



\subsection{Metagenomic Species Pan-genomes}
Microbial pan-genomes are gene repertoires composed of core genes present in all strains and accessory genes present in only some of them (Medini et al., 2005). In a shotgun metagenomic sequencing context, we define as shared the genes detected in some samples where the species is not present.

A strain found in a sample is an instance of the species pan-genome: it is made of all the species (shared) core genes and of a subset of (shared) accessory genes. Core genes are suitable for taxonomic profiling at species-level while accessory genes can be used to compare strains across samples. Genes tagged as shared should be used carefully as they contain false positives counts or are subject to horizontal transfer.
 


	\section{Conclusion}
\begin{itemize}
	\item \emph{New and open areas of research in which the application of metagenomic pipelines are relevant}
	\item \emph{HMP and other }
	\item \emph{The increased impact of machine learning in analysis}
	\item \emph{Short section - just for past-present-future completeness}
	\item \emph{Future developments for metagenomic analysis}
\end{itemize}

%\section{Notes}
%After collection, the steps involved in preparing the sequencing data for metagenomic analysis are quality control, filtering, and trimming.
%Sequence alignment - Bowtie2, Tophat2, Hisat2 are used to map reads against a database
%Classifying taxonomy and Annotation - Binning
%EBI Metagenomics (MGnify) has doubled the number of publicly available anaysed datasets held within the resource in two years.
%Organisms have similar tetranucleotide frequencies put contigs together into genome
%Viral, environmental, gut, long/short reading, computational,lab resourses etc - deep coverage, how did you recover the sequences, oxford nanopore vs illumina, shotgun vs 16s, number of samples? data preparation before binning, gene orientation, webserver vs local vs supercomuter, competency with the linux environment? sequence coverage, methylation signatures

\bibliography{library}
\end{document}

